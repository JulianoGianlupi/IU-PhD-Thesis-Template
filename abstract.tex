%%%%%%%%%%%%%%%%%%%%%%%%%%%%%%%%%%%%%%%%%%%%%%%%%%%%%%%%%
% Do not edit these lines unless you wish to customize
% the template
%%%%%%%%%%%%%%%%%%%%%%%%%%%%%%%%%%%%%%%%%%%%%%%%%%%%%%%%%
\newgeometry{left=1in}

\begin{center}

\yourName\\
\MakeUppercase{\thesisTitle}

\end{center}

\vspace{1.5\baselineskip}

%Insert your abstract here
This Ph.D. research focuses on two primary areas: investigating human health questions using mechanistic Agent-based models (ABMs) of cells and tissues, and enhancing the field of mechanistic bio-ABM by improving model verification and sharing. ABMs offer a bottom-up approach to studying complex biological systems by capturing individual-level behaviors and interactions. The thesis specifically concentrates on agent-based models of cell tissues, with a particular emphasis on COVID-19 and anti-viral treatment. The research explores the integration of traditional pharmacokinetics-pharmacodynamics (PKPD) or physiologically based pharmacokinetics (PBPK) models with ABMs to gain insights COVID-19 treatment with remdesivir. It investigates factors contributing to the low adoption of remdesivir as a COVID-19 treatment, such as the harsh treatment schedule and potency miss-estimation, with a focus on cell response heterogeneity as a potential cause for miss-estimation. Additionally, the research highlights the importance of model cross-platform portability. Running the same model in different platforms can validate models, and ensure model robustness. However, porting a model is difficult and time consuming, emphasizing the need for a universal modeling description standard for ABMs akin to the Systems Biology Markup Language (SBML). To investigate some of the challenges of developing such a standard, this research creates a model specification translation to translate a PhysiCell model into a CompuCell3D model.

% The thesis also discusses methods for comparing population dynamics, transforming ordinary differential equation (ODE) models into ABMs, and ongoing research on interferon pathways, leishmania replication, and immune recruitment. The researcher has contributed to the field by publishing educational tools showcasing the capabilities of CompuCell3D.



\ifdefined\committeeMemberFourTypedName

\null\hfill \myRule\\
\null\hfill \committeeChairpersonTypedName, \committeeChairpersonPostNominalInitials\\
\null\hfill \myRule\\
\null\hfill \committeeMemberTwoTypedName, \committeeMemberTwoPostNominalInitials\\
\null\hfill \myRule\\
\null\hfill \committeeMemberThreeTypedName, \committeeMemberThreePostNominalInitials\\
\null\hfill \myRule\\
\null\hfill \committeeMemberFourTypedName, \committeeMemberFourPostNominalInitials\\

\ifdefined\committeeMemberFiveTypedName
\null\hfill \myRule\\
\null\hfill \committeeMemberFiveTypedName, \committeeMemberFivePostNominalInitials\\
\fi

\fi
\restoregeometry
