\section{Ki-67 Basic Cycle Improved Division Implementation in CompuCell3D}\label{suplemental:cc3d-ex:ki67-improved}
Note: we have removed the implementation of population statistics, data saving, and plots from the code presented here. In places this would cause an error we have added a \code{pass} statement. The model that comes packaged with PhenoCellPy does include population statistics, data saving, and plots.


\begin{python}
from cc3d.cpp.PlayerPython import *
from cc3d import CompuCellSetup
from cc3d.core.PySteppables import *
from numpy import median, quantile, nan
import sys
sys.path.extend(["C:\\PhenoCellPy"])
import Phenotypes as pheno
def Ki67pos_transition(*args):
    # print(len(args), print(args))
    # args = [cc3d cell volume, phase's target volume, time in phase, phase duration
    return args[0] >= args[1] and args[2] > args[3]

class ConstraintInitializerSteppable(SteppableBasePy):
    def __init__(self, frequency=1):
        SteppableBasePy.__init__(self, frequency)
        self.track_cell_level_scalar_attribute(field_name='phase_index_plus_1', attribute_name='phase_index_plus_1')
        self.target_volume = None
        self.doubling_volume = None
        self.volume_conversion_unit = None

    def start(self):
        side = 10
        self.target_volume = side * side
        self.doubling_volume = 2 * self.target_volume
        x = self.dim.x // 2 - side // 2
        y = self.dim.x // 2 - side // 2
        cell = self.new_cell(self.CELL)
        self.cell_field[x:x + side, y:y + side, 0] = cell
        dt = 5  # 5 min/mcs
        ki67_basic_modified_transition = \ 
            pheno.phenotypes.Ki67Basic(dt=dt,
                transitions_to_next_phase=[None,
                    Ki67pos_transition],
                transitions_to_next_phase_args=[None,
                    [-9, 1, -9, 1]])
        self.volume_conversion_unit = self.target_volume / \
            ki67_basic_modified_transition.current_phase.volume.total
        for cell in self.cell_list:
            cell.targetVolume = self.target_volume
            cell.lambdaVolume = 2.0
            pheno.utils.add_phenotype_to_CC3D_cell(cell, 
                ki67_basic_modified_transition)
            cell.dict["phase_index_plus_1"] = \
                cell.dict["phenotype"].current_phase.index + 1
        self.shared_steppable_vars["constraints"] = self
        
class MitosisSteppable(MitosisSteppableBase):
    def __init__(self, frequency=1):
        MitosisSteppableBase.__init__(self, frequency)
        self.constraint_vars = None
        self.previous_number_cells = 0
        self.plot = True
        self.save = False
        if self.save:
            pass
    def start(self):
        self.constraint_vars = self.shared_steppable_vars["constraints"]
        if self.plot:
            pass
    def step(self, mcs):
        if not mcs and self.plot:
            pass
        elif not mcs % 50 and \ 
            len(self.cell_list) - self.previous_number_cells > 0 \
            and self.plot:
            pass
        cells_to_divide = []
        n_zero = 0
        n_one = 0
        volumes = []
        time_spent_in_0 = []
        time_spent_in_1 = []
        for cell in self.cell_list:
            volumes.append(cell.volume)
            cell.dict["phenotype"].current_phase.simulated_cell_volume = cell.volume
            if cell.dict["phenotype"].current_phase.index == 0:
                n_zero += 1
                time_spent_in_0.append(cell.dict["phenotype"].current_phase.time_in_phase)
            elif cell.dict["phenotype"].current_phase.index == 1:
                n_one += 1
                time_spent_in_1.append(cell.dict["phenotype"].current_phase.time_in_phase)
                # args = [cc3d cell volume, 
                #        doubling colume, 
                #        time in phase, 
                #        phase duration]
                args = [
                    cell.volume,
                    .9 * self.constraint_vars.doubling_volume,  
                    # we use 90\% of the doubling volume because cc3d cells
                    # will always be slightly below their target due to the contact energy
                    cell.dict["phenotype"].current_phase.time_in_phase \ 
                        + cell.dict["phenotype"].dt,
                    cell.dict["phenotype"].current_phase.phase_duration]
                cell.dict["phenotype"].current_phase.
                    transition_to_next_phase_args = \ 
                    args
            changed_phase, should_be_removed, divides = \ 
                cell.dict["phenotype"].time_step_phenotype()
            converted_volume = \ 
                self.constraint_vars.volume_conversion_unit * \
                    cell.dict["phenotype"].current_phase.volume.total
            cell.targetVolume = converted_volume
            if changed_phase:
                cell.dict["phase_index_plus_1"] = \ 
                    cell.dict["phenotype"].current_phase.index + 1
            if divides:
                cells_to_divide.append(cell)
        if self.save or self.plot:
            pass
        for cell in cells_to_divide:
            # self.divide_cell_random_orientation(cell)
            # Other valid options
            # self.divide_cell_orientation_vector_based(cell,1,1,0)
            # self.divide_cell_along_major_axis(cell)
            self.divide_cell_along_minor_axis(cell)

    def update_attributes(self):
        # resetting target volume
        converted_volume = \ 
            self.constraint_vars.volume_conversion_unit * \
                self.parent_cell.dict["phenotype"].current_phase.volume.total
        self.parent_cell.targetVolume = converted_volume
        self.clone_parent_2_child()
        self.parent_cell.dict["phase_index_plus_1"] = \ 
            self.parent_cell.dict["phenotype"].current_phase.index + 1
        self.child_cell.dict["phase_index_plus_1"] = \ 
            self.child_cell.dict["phenotype"].current_phase.index + 1
        self.child_cell.dict["phenotype"].time_in_phenotype = 0
    def on_stop(self):
        self.finish()
    def finish(self):
        if self.save:
            pass
\end{python}


