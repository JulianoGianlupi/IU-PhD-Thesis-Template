The general code sequence for an agent-based model using PhenoCellPy begins with the initialization of the Phenotype and its attachment to the relevant agents at simulation-time 0. Then, at each time-step, the Phenotype time-stepping method should be called for each agent with a Phenotype, and the boolean flags for cell phase change, cell removal from the simulation, and division used as needed. An example pseudo-code of a framework using PhenoCellPy can be found in Algorithm~\ref{algo:sequence:init}


\begin{algorithm}[H]
\SetAlgoLined
    \caption{Pseudo-code showing Phenotype initialization, attachment to agents, and time-stepping}\label{algo:sequence:init}
    % import PhenoCellPy as pcp\;
    dt $\gets$ time step length\;
    t $\gets$ 0\;
    T $\gets$ maximum time step\;
    Phenotype $\gets$ Phenotype(dt=dt, other initialization options)\;
    \For{Agent in List of Agents}{
        Agent Phenotype $\gets$ Phenotype\;
        Other Agent Initialization Tasks\;
    }
    
    \While{$t \leq T$}{
        Pre-Agent-Loop Tasks\;
        \For{Agent in List of Agents}{
            changed phase, removal, divides $\gets$ return from Agent Phenotype Time-Step call\;
            \If{changed phase is True}{
                Phase change tasks\;
            }
            \If{removal is True}{
                Call Modeling Framework's Agent Removal Method on Agent\;
            }
            \If{divides is True}{
                 Call Modeling Framework's Agent Division Method on Agent\;
            }
            Other Agent Tasks\;
        }
        Post-Agent-Loop Tasks\;
        t $\gets$ t + dt\;
    }
\end{algorithm}
