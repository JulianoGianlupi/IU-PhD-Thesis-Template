\chapter{An overview of agent based models}\label{cha:abm}


\textcolor{red}{In the usual cellular environment drag forces dominate and speed, not acceleration, is proportional to the force applied to the object [citations] (there is no inertia). \textit{I.e.,}}

% \begin{equation}\label{eq:abm:cst-v}
%     \dift{|\Vec{v}|} \approx 0
% \end{equation}

\begin{equation}\label{eq:abm:aristotle-said-nothing-wrong}
    \Vec{v} \propto \Vec{F}
\end{equation}


\section{Cellular Potts Model}\label{sec:abm:cpm-history}

The Cellular Potts Model (\textit{CPM})~\cite{graner1992simulation, savill_modelling_1997}, also known as Graner-Glazier–Hogeweg (\textit{GGH}) model, is an extension of the Potts model~\cite{potts1952some}. CPM was created to simulate multi-cellular biological systems~\cite{graner1992simulation}. The Potts model, in turn, is a generalization of the Ising model~\cite{ising1925beitrag}. In CPM energy minimization governs the dynamics of the system.
In the following sections I will go over the historical evolution of \textit{CPM}.

\subsection{The Ising Model}\label{sec:abm:cpm-history:ising}

Ising was interested in modeling and explaining the phase transition of ferromagnetic materials losing their magnetic properties when heated~\cite{ising1925beitrag}. Magnetism arises from the alignment of atoms' spins. He developed a simple model that considers short-range interactions of the spins of atoms in a lattice (\textit{i.e.}, in a crystal structure). In his model each spin ($\sigma$) can be either up (denoted by a value of $+1$) or down (denoted by a value of $-1$). The spin configurations obey a Boltzmann probability (equation~\ref{eq:abm:boltzman-prob}), that means that for a high enough temperature any spin configuration is equally probable.

\begin{equation}\label{eq:abm:boltzman-prob}
P(\{\sigma\}) = \exp\left(-\frac{\mH(\{\sigma\})}{k_B T}\right)\,\,,
\end{equation}
\noindent where $\{\sigma\}$ is the spins configuration, and $\mH$ the Hamiltonian (\textit{i.e.}, the system's total energy).

In the absence of an external magnetic field, the total energy of the spin configuration (the Hamiltonian) in Ising's model~\cite{ising1925beitrag} is

\begin{equation}\label{eq:abm:ising-hamiltonian}
\mH_I = -J\sum_{i=1}^{N}\sum_{j\in\mathcal{N}(i)} \sigma_i\sigma_j\,\,.
\end{equation}

\noindent The first sum of the double summation is done over the lattice sites (\textit{i}) and the second is over the first neighbors of \textit{i} ($j\in\mathcal{N}(i)$). The spin interaction energy is \textit{J}.

Each neighboring opposed spins increases the system's energy by $2 \times J$, while neighboring aligned spins decrease the energy by $J$, therefore the spin-grid will rearrange itself to minimize opposing contacts, creating grains of aligned spins. The ferromagnetic transition happens at a critical temperature ($T_C$), for which all spin configurations are equally likely.

\subsection{The Potts Model}\label{sec:abm:cpm-history:potts}

Potts proposed a generalization of Ising's model~\cite{potts1952some}, instead of using only two values for spin he proposed using $q$ values for spin~\cite{potts1952some}. With Potts' model spin configurations that are not aligned to an arbitrary axis can be achieved. The Hamiltonian for Potts' model is

\begin{equation}\label{eq:abm:potts-hamiltonian}
    \mH_P = J \sum_{i=1}^{N}\sum_{j} ( 1 - \delta(\sigma_i,\sigma_j))\,\,.
\end{equation}

\noindent $\delta(a, b)$ is the Kronecker delta function~\cite{} \textcolor{red}{(do I need to cite the Kronecker delta??)}, if $a=b$ then $\delta(a, b)=1$ otherwise it is 0. The double summation is done in the same manner as before, $J$ still is the spin interaction energy. Having $q$ distinct spins is paramount for the development of the Cellular Potts Model (\textit{CPM}).

The coaligned spin grains tend to be pinned to the grid (\textit{i.e.}, tend to be square) in the Potts model. This is a computational artifact, Holm \textit{et al.}~\cite{holm1991effects} showed that to minimize this pinning a greater neighborhood must be used in the second sum of Equation~\ref{eq:abm:potts-hamiltonian}, breaking with the short range assumption used by Potts and Ising.

\subsection{The Cellular Potts Model}\label{sec:abm:apm-history:cpm}

François Graner and James Glazier developed the first iteration of Cellular Potts to test the hypothesis of cell sorting by differential adhesion~\cite{graner1992simulation}. Hogeweg later expanded on CPM by adding a diffusion model to CPM~\cite{savill_modelling_1997}. Another name for CPM is Graner-Glazier-Hogeweg (GGH) model.

Graner and Glazier defined regions of "aligned spin" as being a cell. This means that no two regions can have the same "spin" $q$, as mater of fact, what we called spin so far is from now on the cell-ID (still denoted by $\sigma$). In other words, before each point on the grid represented an atom with a spin value, now regions (comprised of several points on the grid) with the same cell-ID represent a cell.

In CPM cells can be of different types, and the contact energy between cells will depend on both types. This allowed Graner and Glazier to probe the differential adhesion hypothesis~\cite{graner1992simulation}. More energy terms are needed to fully describe a cell, such as an energy related to the cell volume, chemotaxis, cell elongation, external field (\textit{i.e.} gravity), \textit{etc}. The simplest Hamiltonian for CPM is the one defined in~\cite{graner1992simulation}, it is

\begin{equation}\label{eq:abm:cpm-hamiltonian}
\mH_{CPM} = \sum_i \sum_{j_v} J(\tau(\sigma_i),\tau(\sigma_j))(1 - \delta(\sigma_i,\sigma_j)) + \sum_{\sigma}\lambda_V(\sigma)\left(V(\sigma) - V_{TG}(\sigma)\right)^2\,\,.
\end{equation}

\noindent The first term is the contact energy and the second the volume energy. $\sigma$ represent a cell (by referencing its ID), $\tau(\sigma)$ is the type of cell $\sigma$, $J(\tau(\sigma_i),\tau(\sigma_j))$ is the (type-pair dependent) contact energy between two cells, the contact energy is symmetric, \textit{i.e.}, $J(\tau,\tau') = J(\tau',\tau)$. $V(\sigma)$ is the current volume of cell $\sigma$, $V_{TG}(\sigma)$ is the target volume of cell $\sigma$, and $\lambda_V(\sigma)$ is the volume energy intensity for cell $\sigma$. 

The double sum of the contact energy is made over a pixel $i$ and its neighbors $j_v$ (using a higher neighborhood order, as proposed by Holm \textit{et. al}~\cite{holm1991effects}). The Kronecker delta function here guaranties that a cell will not have (external) contact with itself. The volume energy is the harmonic potential, if a cell is above or below its target volume it will change its volume to be closer to the target. The sum for the target volume is made over all cells.

\subsection{Temporal dynamics of Cellular Potts Model}\label{sec:abm:apm-history:cpm-dyn}

Solving the equations for CPM analytically is impossible. Exploring all configurations numerically is also not doable before the heat death of the universe. The algorithm that evolves a CPM is

\begin{enumerate}
    \item Pick a random pixel from the grid ($i$)
    \item Pick another random pixel ($j$) from $i$'s neighbors
    \item If $i$ and $j$ don't belong to the same cell, propose \textcolor{red}{(propose is not the right word, I can only think of this phrase in portuguese...)} that the cell that occupies $i$ moves to occupy $j$ as well. Think as the cell that occupies $i$ is moving and pushing towards the cell that occupies $j$. 
    \item Calculate the change of the system's energy due to the cell movement
    
    \begin{equation}\label{eq:abm:flip-attempt:delta}
        \Delta \mH = \mH(\text{with movement}) - \mH(\text{without movement})
    \end{equation}
    
    \item $\Delta \mH$ then defines a probability of accepting the move or not. It is
    
    \begin{equation}\label{eq:abm:flip-attempt:prob}
        p(\text{accept move}) = \begin{cases}
        1\text{, if } \Delta \mH \leq 0\\
        \exp{\left(-\frac{\Delta \mH}{T}\right)}\text{, otherwise}
        \end{cases}
    \end{equation}
\end{enumerate}

This process is usually called a flip-copy attempt, I call it a move-attempt. In CPM, for a grid with \textit{N} pixels, one time-step is defined as \textit{N} move-attempts. The usual name, for historical reasons, for the time-step in CPM is "Monte Carlo Step" (MCS). $T$ in equation~\ref{eq:abm:flip-attempt:prob} is, also for historical reasons, called temperature. However, it has nothing to do with a temperature one would measure with a thermometer, it's better to think of it as the energy fluctuation amplitude of the cells' borders. \textcolor{red}{more on the temperature?}
The CPM dynamics algorithm leads to a dampened movement in accordance with the equation of movement shown in Equation~\ref{eq:abm:aristotle-said-nothing-wrong}. 

\section{Center Based Models}\label{sec:abm:center}

% https://link.springer.com/article/10.1007/s40571-015-0082-3#Sec7

\pscs is a center-based modelling (\textit{CBM}) framework. In CBMs cells are represented as a particle~\cite{van_liedekerke_simulating_2015} or a group of particles~\cite{teixeira_single_2021}, the particle is, usually, spherical. Unlike CPM, in center based models, forces are modeled explicitly, special consideration must be taken to respect the dominance of drag-forces. \pscs uses different time-steps for calculating the solution of diffusing elements in the simulation, for calculating the cell mechanics, and for performing cell behaviors, $\Delta t_{diff}$, $\Delta t_{mech}$, and $\Delta t_{cells}$ respectively. By default their values are $\Delta t_{diff} = 0.01 min$, $\Delta t_{mech} = 0.1 min$, and $\Delta t_{cells} = 6 min$.


\subsection{Velocity in \psc}
\psc updates each cell velocity ($v_i$) according to the force applied to the cell, obeying Equation~\ref{eq:abm:aristotle-said-nothing-wrong}. The velocity of a cell in \pscs is give by:

\begin{subequations}\label{eq:abm:physi-vel}
    \begin{equation}\label{eq:abm:physi-vel:vel}
        \Vec{v_i} = \sum_{j\in\mathcal{N}(i)} \left[-\Vec{\mathcal{A}}(i, j) -\Vec{\mathcal{R}}(i, j)\right] - \Vec{\mathcal{A}}_B(i) - \Vec{\mathcal{R}}_B(i) + \Vec{v}^{\,\,i}_{mot}\,\,,
    \end{equation}
    \begin{equation}\label{eq:abm:physi-vel:cc-ad}
        \Vec{\mathcal{A}}(i, j) = \sqrt{c_a^i c_a^j}\,\,\nabla\phi(\Delta\Vec{x}(i, j), R_A^i+R_A^j)\,\,,
    \end{equation}
    \begin{equation}\label{eq:abm:physi-vel:cc-rep}
        \Vec{\mathcal{R}}(i, j) = \sqrt{c_r^i c_r^j}\,\, \nabla\psi(\Delta\Vec{x}(i, j), R_R^i+R_R^j)\,\,,
    \end{equation}
    \begin{equation}\label{eq:abm:physi-vel:cb-ad}
        \Vec{\mathcal{A}}_B(i) = c_{a,b}^i \,\,\nabla\phi(-\Vec{d}(x_i), R_A^i)\,\,,
    \end{equation}
    \begin{equation}\label{eq:abm:physi-vel:cb-rep}
        \Vec{\mathcal{R}}_B(i) = c_{r,b}^i \,\,\nabla\psi(-\Vec{d}(x_i), R_R^i)\,\,.
    \end{equation}
\end{subequations}

\noindent Where $i$ is the $i$th cell, $j\in\mathcal{N}(i)$ the neighbors of $i$, $\Vec{\mathcal{A}}(i, j)$ is the adhesion force between cells $i$ and $j$, $\Vec{\mathcal{R}}(i, j)$ the repulsion between cells $i$ and $j$, $\Vec{\mathcal{A}}_B(i)$ the adhesion of cell $i$ to the basement membrane, and $\Vec{\mathcal{R}}_B(i)$ the repulsion of cell $i$ to the basement membrane~\cite{ghaffarizadeh_physicell_2018}. $c_a^i$ and $c_{a,b}^i$ are, respectively, the cell's cell-cell and cell-basement adhesion parameters,  $c_r^i$ and $c_{r,b}^i$ are, respectively, the cell's cell-cell and cell-basement repulsion parameters~\cite{ghaffarizadeh_physicell_2018}. $\Delta\Vec{x}(i, j)$ is the distance of cell $i$ to cell $j$ (\textit{i.e.}, $\Delta\Vec{x}(i, j) = \Vec{x}_i - \Vec{x}_j$), and $\Vec{d}(x_i)$ the distance from the closest point of the basement membrane to the cell~\cite{ghaffarizadeh_physicell_2018}. 

$\phi(\Vec{x}, R)$ is the adhesion energy potential, and $\psi(\Vec{x}, R)$ the repulsion energy potential, they are  zero for $|\Vec{x}|>0$. The form of $\nabla\phi(\Vec{x}, R)$ and $\nabla\psi(\Vec{x}, R)$ are given in \psc's supplemental materials~\cite{ghaffarizadeh_physicell_2018}, they are, respectively:

\begin{equation}\label{eq:abm:physi-ad-pot}
    \nabla\phi(\Vec{x}, R) = \begin{cases}
        \left(1 - \frac{|\Vec{x}|}{R} \right)\frac{\Vec{x}}{|\Vec{x}|}\,\,\,\text{    if }|\Vec{x}|\leq R\\
        0\,\,\,\text{    otherwise}
    \end{cases}\,\,,
\end{equation}
\noindent and,

\begin{equation}\label{eq:abm:physi-rep-pot}
    \nabla\psi(\Vec{x}, R) = \begin{cases}
        -\left(1 - \frac{|\Vec{x}|}{R} \right)\frac{\Vec{x}}{|\Vec{x}|}\,\,\,\text{    if }|\Vec{x}|\leq R\\
        0\,\,\,\text{    otherwise}
    \end{cases}\,\,.
\end{equation}

Finally, $\Vec{v}^{\,\,i}_{mot}$ is cell $i$'s motility. Motility, in \psc, is the persistent motion of the cell~\cite{ghaffarizadeh_physicell_2018}. At each time-step a cell in \pscs has a probability of changing the motility direction given by

\begin{equation}\label{eq:abm:physi:change-vmot}
    Pr(\text{new}\,\,\,\Vec{v}^{\,\,i}_{mot}) = \Delta t_{mech} / \tau^i_{per}\,\,.
\end{equation}
\noindent Where $\tau_{per}$ is the persistence time~\cite{ghaffarizadeh_physicell_2018}. If a velocity update occurs, it will follow

\begin{equation}
    \Vec{v}^{\,\,i}_{mot} = s^i_{mot} \frac{\left(1 - b\right) \hat{\xi} + b \hat{d}^{\,\,i}_{bias}}{|\left(1 - b\right) \hat{\xi} + b \hat{d}^{\,\,i}_{bias}|}\,\,.
\end{equation}
\noindent With $s^i_{mot}$ the cell speed, $\hat{\xi}$ a random unit vector, $\hat{d}^{\,\,i}_{bias}$ the bias direction unit vector, and $b$ the motility randomness amount~\cite{ghaffarizadeh_physicell_2018}. $\hat{d}^{\,\,i}_{bias}$ is determined by the cell state, chemical field (chemotaxis in \pscs is represented as a component of $\hat{d}_{bias}$). The $b$ factor  is $\in[0, 1]$, $b=1$ represents a completely deterministic motility direction along $\hat{d}_{bias}$ and $b=0$ Brownian motion.

