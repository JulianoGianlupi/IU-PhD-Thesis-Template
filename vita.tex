\chapter*{Vita}
\pagenumbering{gobble}
\begin{center}
    Juliano Ferrari Gianlupi
\end{center}
\addtocontents{toc}{
 \unexpanded{\unexpanded{\renewcommand{\cftchapdotsep}{\cftnodots}}}%  
}
\addcontentsline{toc}{chapter}{Curriculum Vitae}

% \doublespacing

\section*{Education}

\subsection*{-- Ph.D. in Intelligent Systems Engineering}
\begin{itemize}\setlength\itemsep{-.9em}
    \item Indiana University Bloomington, USA 
    \item Concentration: Bioengineering
    \item Advisor: James A. Glazier
    \item Duration: Jan 2018 - Aug 2023
    \item Research Areas: agent-based multiscale modeling of cells and tissues, biological dynamical systems, building infrastructure for nanoBIO.
\end{itemize}

\subsection*{-- M.Sc. in Physics}                     
\begin{itemize}\setlength\itemsep{-.9em}
    \item Federal University of Rio Grande do Sul, Brazil
    \item Advisor: Gilberto L. Thomas
    \item Duration: Jan 2016 - Dec 2017
\end{itemize}

\subsection*{-- B.Sc. in Physics}
\begin{itemize}\setlength\itemsep{-.9em}
    \item Federal University of Rio Grande do Sul, Brazil
    \item Duration: Jan 2011 - Dec 2015
\end{itemize}

\section*{  Research Experience}
\subsection*{-- Intelligent Systems Engineering}
\begin{itemize}\setlength\itemsep{-.9em}
    \item Projects: Developing agent-based models of cells to predict tissue-level biological function and disease. Focused on modeling the spread of viral infection in tissues and the cellular immune response. Coupling PK/PKPD simulations with agent-based modeling. Modeling infrastructure development.
    \item Duration: Jan 2018 - Aug 2023
\end{itemize}

\subsection*{-- Eli Lilly \& Co., PK/PD and Pharmacometrics Team}
\begin{itemize}\setlength\itemsep{-.9em}
    \item Project: investigation into a novel way of using available COVID-19 data under the supervision of Dr. Emmanuel Chigutsa, in the PK/PD modeling team.
    \item Duration: May 2022 - Aug 2022
\end{itemize}

\subsection*{-- Federal University of Rio Grande do Sul, Physics}
\begin{itemize}\setlength\itemsep{-.9em}
    \item Project: soft-materials models to investigate the evolution of dry and wet foams, and how the evolution changes with the liquid fraction
    \item Duration: Aug 2014 - Dec 2017
\end{itemize}

\section*{  Teaching Experience}
\subsection*{-- Co-Instructor}
\begin{itemize}\setlength\itemsep{-.9em}
    \item Indiana University Bloomington, Luddy SICE-ISE
    \item Jan 2020 - May 2020; Jan 2021 - May 2023
    \item Courses: Computational Modeling Methods for Virtual Tissues, Introduction to Computational Bioengineering—Dynamics on Networks
\end{itemize}

\subsection*{-- Workshop Instructor}
\begin{itemize}\setlength\itemsep{-.9em}
    \item 2021 Virtual CompuCell3D User Training Workshop
    \item 2020 Virtual CompuCell3D User Training Workshop
    \item 2019 Virtual CompuCell3D User Training Workshop
    \item 2018 Virtual CompuCell3D User Training Workshop
    \item 2017 Virtual CompuCell3D User Training Workshop
\end{itemize}

\section*{  Publications}
\subsection*{-- Journal Articles}

\begin{itemize}\setlength\itemsep{-0.9em}
    \item \underline{Gianlupi, J. F.}, Sego, T. J., Sluka, J. P., \& Glazier, J. A. (2023). PhenoCellPy: A Python package for biological cell behavior modeling. bioRxiv, 2023-04.
    \item \underline{Ferrari Gianlupi, J.}, Mapder, T., Sego, T. J., Sluka, J. P., Quinney, S. K., Craig, M., ... \& Glazier, J. A. (2022). Multiscale model of antiviral timing, potency, and heterogeneity effects on an epithelial tissue patch infected by SARS-CoV-2. Viruses, 14(3), 605.
    \item Sego, T. J., Aponte-Serrano, J. O., \underline{Gianlupi, J. F.}, \& Glazier, J. A. (2021). Generation of multicellular spatiotemporal models of population dynamics from ordinary differential equations, with applications in viral infection. BMC biology, 19(1), 1-24.
    \item Zarnitsyna, V. I., \underline{Gianlupi, J. F.}, Hagar, A., Sego, T. J., \& Glazier, J. A. (2021). Advancing therapies for viral infections using mechanistic computational models of the dynamic interplay between the virus and host immune response. Current Opinion in Virology, 50, 103-109.
    \item Sego, T. J., Aponte-Serrano, J. O., \underline{Ferrari Gianlupi, J.}, Heaps, S. R., Breithaupt, K., Brusch, L., ... \& Glazier, J. A. (2020). A modular framework for multiscale, multicellular, spatiotemporal modeling of acute primary viral infection and immune response in epithelial tissues and its application to drug therapy timing and effectiveness. PLoS computational biology, 16(12), e1008451.
    \item de Lima, C. F., \underline{Gianlupi, J. F.}, Metzcar, J., \& Zerick, J. (2020). Accelerated solving of coupled, non-linear ODEs through LSTM-AI. arXiv preprint arXiv:2009.08278.
    \item Getz, Michael, \underline{et al.} "Iterative community-driven development of a SARS-CoV-2 tissue simulator." BioRxiv (2020): 2020-04.
\end{itemize}

\subsection*{-- Conferences}
\begin{itemize}\setlength\itemsep{-0.9em}
    \item APS 2023 March Meeting, March 2023, contributed talk
    \item 12th European Conference on Mathematical and Theoretical Biology, September 2022, contributed  talk
    \item German Conference on Bioinformatics, September 2022, contributed  talk
    \item IMAG/MSM Working Group (Multiscale Modeling and Viral Pandemics), February 2022, invited talk
    \item 2020 Society of Mathematical Biology, August 2020, Contributed Poster
\end{itemize}


\subsection*{-- Computational Tools}
\begin{itemize}\setlength\itemsep{-0.9em}
    \item \textbf{Script for deploying CompuCell3D simulations as tools in nanoHUB}: Script to make deployment of \ccds simulations on nanoHUB, \url{https://github.com/JulianoGianlupi/cc3d-nanoHub-tool-maker}
    \item \textbf{CompuCell3D v4 Main Tool}: Base tool for CompuCell3D version 4 and greater. Includes running any of the demos included with CC3D, \url{https://nanohub.org/tools/cc3dbase4x}
    \item \textbf{CompuCell3D - Bacterium Macrophage}: online deployment of \ccds simulation of  Macrophage hunting bacterium through a maze, \url{https://nanohub.org/tools/cc3dbactmacro}
    \item \textbf{CompuCell3D - Simulation of cell crawling in 3D}: Online deployment of CompuCell3D simulation of cell crawling in 3D by Fortuna et al. 2020 \url{https://doi.org/10.1016/j.bpj.2020.04.024}, \url{https://nanohub.org/tools/gltcellcrawl}
    \item \textbf{CompuCell3D - 2D wet foam coarsening}: CompuCell3D Demo for a 2D foam coarsening without drainage, \url{https://nanohub.org/tools/cc3dwf}
    \item \textbf{CompuCell3D - 2D wet foam coarsening with drainage}: CompuCell3D Demo for a 2D foam coarsening with drainage, \url{https://nanohub.org/tools/cc3dwfdrain}
    \item \textbf{CompuCell3D v4 - Bacterium Macrophage simulation}: Macrophage hunting bacterium through a maze using CompuCell3D v4, \url{https://nanohub.org/tools/bacmacrocc3d4}
    \item \textbf{CompuCell3D - Simulation of angiogenesis}: CompuCell3D simulates angiogenesis using chemical signals, \url{https://nanohub.org/tools/angiogencc3d}
    \item \textbf{CompuCell3D - Delta-Notch signaling in a group of cells}: CompuCell3D can solve individual cell's ODE and have the information of one cell affect another (implemented trough SBML), \url{https://nanohub.org/tools/deltanotchcc3d}
    \item \textbf{CompuCell3D - Cells random walking at different speeds}: Cells random walking at different speeds implemented through the motility plugin, \url{https://nanohub.org/tools/mot2ddemocc3d}
    \item \textbf{CompuCell3d cell sorting in a hexagonal lattice}: Showcases hexagonal lattice use in CompuCell3D by simulating cell sorting by difference in contact energies in a hexagonal lattice, \url{https://nanohub.org/tools/cellsorthexcc3d}
    \item \textbf{COVID 19 Virtual Tissue Model - Tissue Infection and Immune Response Dynamics}: Simulates tissue and immune system interactions during a viral lung infection, \url{https://nanohub.org/tools/cc3dcovid19}
    \item \textbf{Cancer Evolution in CompuCell3D}: Demonstrates implemetation of cancer evolution in CompuCell3D, \url{https://nanohub.org/tools/cancerevocc3d}
    \item \textbf{FocalPointPlasticity Plugin Demo}: This simple demo shows basic functionality of FocalPointPlasticity and how link rigidity affects cell behaviors, \url{https://nanohub.org/tools/cc3dfppdemo}
    \item \textbf{CompuCell3D - Chemotactic Elongation Demo}: Implementation of Developmental biology 289.1 (2006): 44-54. and PLoS Comput Biol 4.9 (2008): e1000163, \url{https://nanohub.org/tools/cc3delongdemo}
    \item \textbf{COVID-19 drug treatments explorer, CompuCell3D}: Explores possible drug treatments for COVID-19; namelly viral entry inhibition and viral replication inhibition, \url{https://nanohub.org/tools/coviddrugexp}
    \item \textbf{CompuCell3D Vascular Tumor}: Simulate 3D vascular tumor with CompuCell3D, \url{https://nanohub.org/tools/cc3dvasctumor}
    \item \textbf{CompuCell3D - Avascular Tumor Growth and Mutation}: vascular tumor, growing on nutrient, self-limited, mutates, \url{https://nanohub.org/tools/avasctum}
\end{itemize}



\section*{Campus Activities}
\subsection*{-- Indiana Graduate Workers Coalition: Union Representative}
\begin{itemize}\setlength\itemsep{-.9em}
    \item Jan 2022 - Dec 2022
\end{itemize}

\subsection*{-- Indiana Graduate Workers Coalition: Social Media Manager}
\begin{itemize}\setlength\itemsep{-.9em}
    \item Jan 2022 - Dec 2022
\end{itemize}

\section*{Languages}
\begin{itemize}\setlength\itemsep{-.9em}
    \item Portuguese, native
    \item English, fluent
    \item Spanish, advanced
\end{itemize}

\section*{Skills}
\begin{multicols}{3}
\begin{itemize}\setlength\itemsep{-.9em}
    \item Python
    \item R
    \item NONMEM
    \item COPASI
    \item Fortran 90
    \item C++
    \item PyThorch
    \item TensorFlow
    \item \ccds
    \item \pscs
    \item Git
    \item \LaTeX
\end{itemize}
\end{multicols}


% \pagenumbering{gobble}