\chapter{Introduction and Background}\label{cha:intro}

My Ph.D. research has two areas of focus. The first is investigating questions related to human health using mechanistic Agent-based models (ABMs) of cells and tissues. The second is strengthening the field of mechanistic bio-ABM by investigating how to make models easier to be verified and shared. 

% ~\cite{10.1371/journal.pone.0033726, ozik_high-throughput_2018} to embryonic development~\cite{graner1992simulation,hester2011multi,adhyapok2021mechanical} and pathogen infections~\cite{sego_generation_2021,ferrari_gianlupi_multiscale_2022,sego_modular_2020,sego2022multiscale}

Agent-based models are powerful tools for investigating complex biological systems~\cite{sego_modular_2020, ferrari_gianlupi_multiscale_2022, sego_generation_2021, sego2022multiscale, 10.1371/journal.pone.0033726, ozik_high-throughput_2018, graner1992simulation,hester2011multi,adhyapok2021mechanical}, offering a bottom-up approach that captures individual-level behaviors and interactions.  This thesis aims to contribute to the advancement of mechanistic modeling in human health by focusing on agent-based models of cell tissues, with a particular emphasis on COVID-19 and anti-viral treatment for it. In Chapters~\ref{cha:sego-aponte} and~\ref{cha:remdes} I present my work on COVID-19 models (published in~\cite{sego_modular_2020} and~\cite{ferrari_gianlupi_multiscale_2022}).
% This research study focuses on the development and application of mechanistic models for human health using agent-based modeling techniques. 

My research explores the dynamics of COVID-19 within a patch of the lung. It investigates how can more traditional pharmacokinetics-pharmacodynamics (PKPD) or physiologically based pharmacokinetics (PBPK) models can be integrated with  ABMs. A key objective is to examine the insights that can be gained from the combination of traditional pharmacometrics models and ABMs, particularly in relation to the low adoption of remdesivir as a COVID-19 treatment. The research investigates potential factors for the low adoption: such as the harsh treatment schedule and drug potency miss-estimation. It identifies response heterogeneity of cells as a possible cause for the potency miss-estimation. Probing response heterogeneity by cells can't be done with more traditional PBPK. 

In science, our experiments (models, simulations, in the case of computational science) should be replicatable and reproducible using a different method, which provides a means of comparing results and ensuring model robustness. Tissue agent-based models lack these qualities, causing a crisis of reproducibility. Reproduction can be done if the model is re-implemented by hand in a different platform. However, the process of cross-platform validation is time-consuming and challenging, requiring the translation and adaptation of model specifications from one framework to another. Besides being an imperfect application of the scientific method, the lack of these qualities hinders collaboration and causes a reproducibility crisis.

My work encompasses the creation of a method to bridge the replication gap between two modeling platforms: CompuCell3D~\cite{swat_multi-scale_2012} and PhysiCell~\cite{ghaffarizadeh_physicell_2018}. I created a translator to go from the model specification of one platform to the specification of another, see Chapters~\ref{cha:translator} (unpublished). In the future, the field of tissue ABMs can be made stronger by the creation of a universal modeling description standard akin to the Systems Biology Markup Language (SBML)~\cite{hucka_systems_2003} for ABMs. My translator is a step in the direction of the universal model spec, as it investigates possible pitfalls and find solutions to som eof them.

Another issue with replication is the miss-match of concepts. If platform A has a model of, \textit{e.g.}, cell shape and platform B does not, we need to either make the cell shape model available to platform B, or decide what to do about the missing concept. In Chapter~\ref{cha:phenocell} (published in~\cite{gianlupi_phenocellpy_2023}) I present my work on making PhysiCell's cell phenotype sub-models available to any Python-based modeling platform by re-imagining, and re-implementing, them in pure Python.

% In Chapters~\ref{cha:translator} (unpublished) and~\ref{cha:phenocell} (published in~\cite{gianlupi_phenocellpy_2023}) I will discuss my work on bridging different model methodologies for Agent Based Models.

% Model validation is of the utmost importance for science. One method to validate ABMs is by running them on different platforms, which provides a means of comparing results and ensuring model robustness. However, the process of cross-platform validation is time-consuming and challenging, requiring the translation and adaptation of model specifications from one framework to another. The variations in modeling approaches and biological representations between these frameworks highlight the need for a universal modeling description standard akin to the Systems Biology Markup Language (SBML)~\cite{hucka_systems_2003} for ABMs. In Chapters~\ref{cha:translator} (unpublished) and~\ref{cha:phenocell} (published in~\cite{gianlupi_phenocellpy_2023}) I will discuss my work on bridging different model methodologies for Agent Based Models.

Besides the work presented in this thesis, I have established methods to compare population dynamics, ordinary differential equation (ODE), models with ABMs, and how to transform an ODE model into an ABM~\cite{sego_generation_2021}. I have ongoing work investigating interferon pathways in cellular innate immune response, and ongoing work into the replication of leishmania in the infection site and immune recruitment to the infection site. I have also published several online educational tools to demonstrate \ccd's capabilities.

 % This thesis aims to contribute to the advancement of mechanistic modeling in human health by focusing on agent-based models of cell tissues, with a particular emphasis on COVID-19 and anti-viral treatment. In Chapters~\ref{cha:sego-aponte} and~\ref{cha:remdes} I present my work on COVID-19 models (published in~\cite{sego_modular_2020} and~\cite{ferrari_gianlupi_multiscale_2022}), and in Chapters~\ref{cha:translator} (unpublished) and~\ref{cha:phenocell} (published in~\cite{gianlupi_phenocellpy_2023}) my work on bridging different model methodologies for Agent Based Models (\textit{ABM}s). Besides the work presented in this thesis, I have established methods to compare population dynamics, ordinary differential equation (ODE), models with ABMs, and how to transform an ODE model into an ABM~\cite{sego_generation_2021}. I also have ongoing work investigating interferon pathways in cellular inate immune response, and ongoing work into the replication of leishmania in the infection site and immune recruitment to the infection site. %TODO: Mention the review somehow?}

%  Digital twins (\textit{DT}) are computer models of objects that merge template descriptions of general examples of the object class with specific data on individual instances of that object. E.g. a template model of an airplane engine or a human heart and data describing the function and properties of a specific engine or my own heart, neither of which are identical to other instances of that class. They are used in monitoring and control of complex systems, from autonomously flying drones to self-diagnosing airplane engines and oil pipelines~\cite{boschert_digital_2016, tuegel_reengineering_2011}. They allow for predictive care of those systems, reducing costs of operation and failures. 
 
% These digital twins are powered through the use of computer models, for instance, stress-strain simulations in an airplane engine, paired with a heat transfer model of that engine, together with other models of physical processes that the engine is subjected to. When available, these methods use real-world continuously available data from the object they are modeling. 
% Recently artificial intelligence/machine learning (\textit{AI/ML}) methods are used to enhance \textit{DT}s (\textit{e.g.}, image analysis, parameter identification, computation acceleration).

% Why aren't similar systems being used extensively in medicine? In short, because biology is a series of interconnected, complex systems, and we do not have a good mechanistic understanding of it. 

% Because of this, in current medical practice, beyond general advice like “don’t smoke” or “eat a healthy diet,” most treatments are reactive, only being sought and used after a problem occurs instead of being proactive, avoiding the health issues altogether. Our inability to replicate treatments and controls for individuals as well as the way treatments are evaluated and approved means that they are optimized for often untypical “typical” patients rather than the actual person receiving the treatment. So far precision or personalized medicine has not radically improved therapies for most patients. They also have not shifted treatment from reactive to proactive~\cite{joyner_promises_2019, szabo_opinion_nodate}.

% The digital health twin will be a constantly evolving self-improving dynamic model of a person’s health, allowing for predictive medicine. With predictive medicine, we could identify whether it’s probable an individual has problems (diagnosis) that have not manifested yet, predict how their health will change, whether by the onset of a disease or the progression of existing disease (prognosis) and be able to quickly evaluate treatment options and evaluate outcome probabilities both for chronic conditions like diabetes or aging and for acute conditions like viral infection or trauma (treatment optimization). This, in turn, would allow treatment fine-tuning, minimizing bad outcomes or undesired side effects. Human health Digital twins will be game-changing tools, mitigating complex problems in biomedicine and clinical decision-making.

% Currently, there are many attempts at using AI and ML models to predict drug effectiveness~\cite{chang_ai-driven_2022} or how a disease will progress in a patient~\cite{schmidt-erfurth_prediction_2018, he_deep_2016, ahn_application_2021}. One usual struggle of these types of models is that the training of AI and ML models is data-hungry, AI/ML methods have thousands or even millions of hidden variables~\cite{he_deep_2016}, but in some cases, there are only hundreds of data points available~\cite{peng_empirical_2020}. AI/ML methods having this many variables without enough data is a recipe for an overfit model that does not replicate and forecasts reality and is not generalizable~\cite{charilaou_machine_2022, noauthor_understanding_nodate}, but can only recognize a result as a part of its training set. %( TODO {more on this?}) 
% Lack of data will be a constant problem for biological machine learning and AI models, as biological data is expensive to obtain, takes a long time to be obtained (in the case of drug development, it takes years), and is often sparse (it’s usual for data points to be temporarily separated by days or weeks). Some of these issues cannot be solved in the wet-lab while respecting both animal and human rights related to research. 

%  One way to bridge the data gap for AI/ML, and simultaneously deepen our understanding of human health and biology, is to create \textit{in silico} (computer) models of biological systems. These models have, by themselves, exploratory and explanatory power. They can also be used, after being well-validated, to generate synthetic data for the training of AI/ML models. Generation and use of synthetic data for \textit{AI/ML} training and testing is an established method~\cite{kim2022transferable, nikolenko2019synthetic}.

%  To make digital twins a for human health a reality, therefore, we must create new mechanistic models of human biological systems. For instance,we need to characterize the immune system and how it reacts and is affected by infections, and how treatments affect both the immune system and the infection. We also need to guarantee that the models we build are well validated. One way to do the validation is to use different methodologies and platforms to cross-validate the model. However, rebuilding a model for different platforms and methods is time-consuming and hard. 
 
 % This thesis aims to contribute to the advancement of mechanistic modeling in human health by focusing on agent-based models of cell tissues, with a particular emphasis on COVID-19 and anti-viral treatment. In Chapters~\ref{cha:sego-aponte} and~\ref{cha:remdes} I present my work on COVID-19 models (published in~\cite{sego_modular_2020} and~\cite{ferrari_gianlupi_multiscale_2022}), and in Chapters~\ref{cha:translator} (unpublished) and~\ref{cha:phenocell} (published in~\cite{gianlupi_phenocellpy_2023}) my work on bridging different model methodologies for Agent Based Models (\textit{ABM}s). Besides the work presented in this thesis, I have established methods to compare population dynamics, ordinary differential equation (ODE), models with ABMs, and how to transform an ODE model into an ABM~\cite{sego_generation_2021}. I also have ongoing work investigating interferon pathways in cellular inate immune response, and ongoing work into the replication of leishmania in the infection site and immune recruitment to the infection site. %TODO: Mention the review somehow?}


\section{Modeling SARS-CoV-2 Infection and Treatment}\label{sec:intro:covid}
The COVID-19 pandemic has underscored the need for comprehensive modeling approaches to better understand the disease dynamics, transmission patterns, and the effectiveness of interventions. Agent-based models, capable of capturing the spatial and temporal aspects of cell behavior, offer a unique opportunity to simulate the complex interactions between immune cells, viral particles, and therapeutic agents within tissues. By incorporating detailed biological mechanisms and experimental data, these models have the potential to provide valuable insights into the underlying processes of COVID-19 and aid in the development of targeted treatment strategies.

In this thesis I explain our first SARS-COV-2 model~\cite{sego_modular_2020} (see Chapter~\ref{cha:sego-aponte}). We aimed to replicate the reported infection time-line and multiple infection outcomes. We investigated the interplay of viral infectivity and immune response intensity to question if those two parameters are enough to replicate the wide-array of outcomes. We only had palliative treatments available at the time, some people recovered or died quickly, while some were sick for weeks. We also did a preliminary implementation of antiviral treatment, that I expanded in~\cite{ferrari_gianlupi_multiscale_2022} (see Chapter~\ref{cha:remdes}).

In my COVID-19 remdesivir treatment work~\cite{ferrari_gianlupi_multiscale_2022} (see Chapter~\ref{cha:remdes}), I investigate how to integrate a physiologically based pharmacokinetics (\textit{PBPK}) and pharmacodynamics (\textit{PD}) model with an ABM, and what other questions can this pairing ask. I investigated what are the effects of cell individuality on the overall treatment, this is a pertinent question because remdesivir is a prodrug that has to be metabolized intracellularlly into its active form~\cite{eastman_remdesivir_2020}.

The COVID-19 pandemic has highlighted the urgent need for efficient model development and collaborative efforts among research groups to gain valuable insights into the complex dynamics of the disease. Furthermore, the ability to compare and validate models is crucial for ensuring their reliability and applicability. A promising approach to model validation involves implementing the same underlying biological processes in different computational frameworks. However, the process of re-implementing a model in a new platform is often challenging and time-consuming, posing a significant barrier to cross-platform validation. My thesis begins to address this difficulty.


\section{Cross-Platform Validation}\label{sec:intro:cross}
Agent-based modeling is a powerful tool for studying complex biological systems, offering a unique perspective on the dynamics and interactions of individual entities within a population. This approach has shown great promise for the investigation of the mechanics of tissue behaviors, from cancer~\cite{10.1371/journal.pone.0033726, ozik_high-throughput_2018} to embryonic development~\cite{graner1992simulation,hester2011multi,adhyapok2021mechanical} and pathogen infections~\cite{sego_generation_2021,ferrari_gianlupi_multiscale_2022,sego_modular_2020,sego2022multiscale}. They enable the investigation of intricate cellular behaviors and their impact on disease progression, treatment strategies, and therapeutic interventions. 

One crucial aspect of advancing agent-based modeling in human health is the cross-platform validation of models. Model validation plays a pivotal role in ensuring the reliability and predictive accuracy of simulations. However, due to the diversity of modeling frameworks and methodologies, validating agent-based models across different platforms presents significant challenges. This thesis addresses this issue by proposing a prototype method for cross-platform validation of agent-based models of cell tissues, bridging the gap between various modeling paradigms. 

Cellular Potts models (CPMs) and center-based models (CBMs) are two commonly employed approaches in agent-based modeling of cell tissues, each with its strengths and limitations.
 There are two possible strategies for bridging the gap: creating a general model description, or by translating model specification from one platform to another. In this thesis I explored the translation of \pscs into \ccd,
 % This thesis focuses on the translation of models built using CPM into CBM, 
exploring the necessary adaptations and overcoming the conceptual and computational differences between the methodologies. By enabling the conversion of models from one methodology to another, this research aims to enhance the accessibility and applicability of agent-based models for studying human health, see Chapter~\ref{cha:translator}. I have also implemented \psc's phenotype submodel~\cite{ghaffarizadeh_physicell_2018} into an independent Python package that can be used by any Python-based modeling platform (see Chapter~\ref{cha:phenocell}).



 