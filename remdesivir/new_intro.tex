The COVID-19 pandemic has inspired the rapid discovery, development, and distribution of antiviral and immune-modulatory drugs and vaccines. Computer simulations of within-host response have assisted in the rapid screening of candidate drug treatments~\cite{keshavarzi_arshadi_artificial_2020}. Mathematical models and their computer simulations enable us to explore alternative treatment regimens using existing drugs rapidly~\cite{jenner_leveraging_2020}. Models of absorption, distribution, metabolism, and elimination (\emph{ADME}) in specific organs and the body as a whole and the pharmacokinetics of drugs within individual cells, such as at the cellular infection level, and the immune system, can be leveraged to advise clinical trials for infectious diseases~\cite{ciupe_modeling_2018, best_mathematical_2018, schiffer_herpes_2018}.


Clinical trials of remdesivir as a possible treatment for COVID-19 followed the declaration of the 2020 pandemic by the World Health Organization~\cite{cao_remdesivir_2020, humeniuk_safety_2020, spinner_effect_2020}. Remdesivir is a single diastereomeric mono-phosphoramidate prodrug designed to arrest the replication of RNA viruses. Upon remdesivir administration, the patient's body generates sequential metabolic intermediates before forming the active nucleoside triphosphate, GS--443902 (GS-441524-triphosphate). The active metabolite then binds to the elongating viral RNA synthesized by RNA-dependent RNA polymerase (\emph{RdRp}) as a nucleoside analog and blocks viral replication~\cite{kokic2021mechanism}. The first clinical trials for remdesivir were as a treatment for the Ebola virus~\cite{eastman_remdesivir_2020, spinner_effect_2020}. Most of these trials administered a 200 mg intravenous (\emph{IV}) infusion loading dose followed by 100 mg IV daily infusions for five to ten days. However, the full breadth of therapeutic schedules remains unexplored, given the urgency required for drug development during the pandemic. To that end, we modelled remdesivir and its mechanism of action (\emph{MOA}) on a patch of lung epithelial tissue infected by SARS-CoV-2 to provide a more comprehensive understanding of the interplay of remdisivir dose and timing, and outcomes. 

Even though we frame our work in the context of SARS-CoV-2 and remdesivir treatment, our methods are general to other viral infections and antiviral treatments. We have developed our own MOA model for remdesivir. There are several antiviral drugs with similar MOAs~\cite{mitja2020use,de2016approved}, and previous modeling works simulated treatment with these drugs~\cite{zitzmann2018mathematical,cao2017mechanisms}. Experiments on SARS-CoV-2 infection in non-human primates (rhesus macaques) and associated mathematical models have shown that an antiviral drug treatment with lower efficacy may elongate the duration of the viremic profile even if the treatment initiation is very early~\cite{kim2021incomplete, dobrovolny_quantifying_2020, williamson2020clinical}. Given these results, the present model focuses on the relations of the drug potency, treatment initiation time and dose interval with the viraemia as crucial players, and performs a thorough scan of all related parameters to elucidate a reasonable treatment regimen.

Various models have described remdesivir's pharmacokinetics (PK), models ranging from one-compartment models to complex physiologically-based pharmacokinetic (\emph{PBPK}) models~\cite{gallo_hybrid_2021, humeniuk_safety_2020, cao_remdesivir_2020}. Researches have also developed combined phar\-ma\-co\-ki\-ne\-tic-phar\-ma\-co\-dy\-na\-mic (\emph{PK-PD}) models for COVID-19. Goyal \emph{et al.} used a two-compartment PK model for remdesivir at different potency and timing of treatment to predict how other parameters affect the disease progression and treatment efficacy~\cite{goyal_potency_2020, goyal_mathematical_2020}. The authors observed that initiating antiviral treatment after symptom onset required antiviral concentrations that reduced viral production rate by more than 90\% ($>$90\% drug efficacy) to achieve a two log reduction in plasma viral load. If administration started at the time of infection (before the onset of symptoms), 60\% drug efficacy achieved a similar reduction of viral load. They have also run theoretical kinetics of remdesivir drug resistance for various treatment regimens.

For the pharmacokinetics of remdesivir, we modified a PBPK model created by Gallo~\cite{gallo_hybrid_2021}, a hybrid, full-PBPK model for remdesivir with 15 tissue compartments. Their PBPK model is very detailed and recovers remdesivir's dynamics in several tissues and plasma. Our focus is on the concentration of the active metabolite of remdesivir in lung epithelial cells; therefore, we opted to simplify Gallo's model (see Methods~\ref{sec:remdes:met:remdes_pk}). As remdesivir is given intravenously (IV) and has a long half-life ($t_{1/2} = 30.2h$)~\cite{humeniuk_safety_2020}, we study dosing intervals longer than one day. Treatments using more extended periods may be helpful for patients that require remdesivir administration but are not in a condition severe enough that requires hospitalization. This approach could help alleviate hospital overcrowding and could improve treatment adherence. We also aim to characterize the interplay of drug potency and schedule on infection dynamics and treatment outcomes.

The above models (coupled population, PK and PD models) can be used to study effects on infection dynamics arising from changing drug potency, half-life, and dosing schedule. Population models assume well-mixed conditions, meaning that the model exposes the entire cell population to the same amount of infectious virus at any instance. A cellular agent-based model (\emph{ABM}) can complement such models by adding multi-cellular-scale resolution~\cite{zarnitsyna2021advancing}. ABMs are an effective simulation technique to model a population of agents. In ABMs the agents are capable of independent decision-making according to assigned attributes and conditions~\cite{glazier2007magnetization, bonabeau2002agent}. Cellular ABMs can introduce tissue heterogeneity to models by their very nature, as cells are individually modeled and can differ from one another, space itself is a model component~\cite{sego_modular_2020, sego2022multiscale, 10.1371/journal.pone.0007190, gast2016computational, getz_iterative_2021}. A recent report on comparative biology immune  ABM (\textit{CBIABM}) has presented a model of mechanism-based differences in bat and human immune systems and discusses the consequences of these differences on disease manifestation~\cite{cockrell2021comparative}. In~\cite{sego2022multiscale}, population models of infection calibrated to experimental data were used to generate an equivalent spatially heterogeneous \emph{ABM} of infection. The authors found that viral infectivity estimates using the ABM differed from the estimates from the population model by as much as 95\%~\cite{sego2022multiscale}. These differences in viral infectivity, or some other characteristic of the infection dynamics, could mean that a population model and an ABM calibrated to the same experimental data can significantly differ in their estimates for effective drug doses and schedules.

Furthermore, infection in a tissue starts from some discrete points of infection and spreads from them~\cite{zeng2020pulmonary, remmelink2020unspecific}. Therefore, spatially heterogeneous distribution of target cell states is expected, with further disease progression near the initial infection location (necrotic sites), to regions farther from initial infection sites, where the infection has not begun. Cytokines concentrations will also be heterogeneous. Since infection can spread within a tissue even if a few cells release virus, this spatial relationship between uninfected and virus releasing cells may determine how effective an antiviral needs to be to contain the viral spread. Heterogeneity in cells reactions and drug delivery and its possible effects on disease and treatment is a topic of active study for COVID-19~\cite{fiege2021single}, other diseases, and substance toxicity~\cite{ordonez2020dynamic, ben2019spatial, diaz2013multi, dheda2018drug}.

In the present ABM, we leverage our already established model of epithelial lung tissue infected by SARS-CoV-2~\cite{sego_modular_2020} implemented in CompuCell3D~\cite{swat_multi-scale_2012}. Our simulated environment models epithelial lung tissue infected by SARS-CoV-2, including cell surface-receptor (\emph{ACE2}) affinity, intra--cellular viral replication, infectious-diffusive virus release, immune response, cytokine signaling by the epithelial and immune cells. We expand on those capabilities by incorporating a pharmacokinetic (PK) model of remdesivir and its dosing regimens, as well as a model for remdesivir's mode-of-action (\emph{MOA}). We explore the effect of varying the time of treatment initiation (from the number of hours after the infection of ten epithelial cells), the potency of remdesivir's active metabolite (by varying $IC_{50}$), and the interval between doses.

As we are simulating a spatially resolved model, we can test the effects of cell-to-cell variability. The amount of drug reaching each cell in the target tissue varies. This variation can result from: (1) different availability and different distance from capillaries (microdosimetry); (2) uptake rate differences (density and dynamics of cell-surface proteins); (3) conversion rate from prodrug to active metabolite based on intra--cellular enzyme concentrations; (4) Effect of cell-ageing on metabolic rates; (5) cell-cycle phases. To model each of these separately one needs a detailed model of cellular metabolism, life cycle and capillary structure. Therefore, in the simulations, we expose each cell to a homogeneous concentration of the antiviral drug, and combine the different possible sources of intra--cellular metabolic heterogeneity into an effective change of the uptake and elimination rates, see Methods~\ref{sec:remdes:met:sego_model} and~\ref{sec:remdes:met:intercell_met_var}.

% (1) differential drug delivery to the cell location, (2) differential uptake of the drug by the cell, and (3) differential elimination of the drug by the cell. The present study does not consider the variability in the drug delivery to the tissue site but does consider the cell-to-cell variability of drug uptake and elimination. 

% In the simulations, we expose each cell to a homogeneous concentration of the antiviral drug, but the intra--cellular metabolism is heterogeneous. See Methods~\ref{sec:met:sego_model} and~\ref{sec:met:intercell_met_var}.

Cells with internal concentrations of remdesivir-active metabolite below concentrations that control the viral replication are significant contributors to viral synthesis and release, and determine the consequent spread of infection. Their spatial distribution in the tissue is key, as those will be the regions of significant infection activity. The duration over which the concentration of the active metabolite is below the effective concentration also matters. Our previous work~\cite{sego_modular_2020} demonstrated that if cells unblock RNA synthesis, even for a short time, the amount of functional RNA produced will be small, and one can expect reasonable inhibition of viral release.

We believe our methods can be of great use in early drug and treatment development when characterization of the drug's PK and PD are not well established. We change the remdesivir's potency and half-life in our model to investigate how those changes affect the disease progression and treatment effectiveness (see Results~\ref{sec:remdes:res:treat:fine} and~\ref{sec:remdes:res:treat:faster}). Our heterogeneous drug metabolism model predicts that higher doses (by $\approx50\%$ or more) are necessary to achieve the same level of treatment success compared to our homogeneous metabolism model results (see Results~\ref{sec:remdes:res:treat:fine} and~\ref{sec:remdes:res:treat:hetero}). Although the cellular level heterogeneity has not been measured experimentally, our results suggests that treatment outcomes depend on the intensity of heterogeneity (see Results~\ref{sec:remdes:res:change-sd}). We hypothesize that the least sensitive cells to the antiviral drive the infection forwards (super-spreader cells).

This work addresses the following questions: How significant are the effects of remdesivir's dosing interval on treatment outcomes? What is the impact of heterogeneous cellular drug uptake and elimination on viral load (heterogeneous cellular drug metabolism)?
