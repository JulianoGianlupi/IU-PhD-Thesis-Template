
%%%%%%%%%%%%%%%%%%%%%%%%%%%%


Here we have the results for the untreated simulations with different initial conditions, namely with 1, 2, 5 and 10 infected cells at the start of the simulation. All of the results for this appendix use 400 simulation replicas for each set of parameters used. For all subfigures the median measurement of simulation replicas is the black line, the 0th to 100th quantiles are shaded as dark blue, 10th to 90th shaded in orange, and 25th to 75th as light blue.

\begin{figure}[H]

  \begin{subfigure}[b]{0.48\linewidth}
    \centering
    \includegraphics[height=0.18\textheight]{remdesivir/sego_var_fig/review/1inf/dead_pop.png} 
    \caption{} 
    \label{fig:res:var_sego:dead_pop:1inf} 
  \end{subfigure} 
  \hspace{\fill}  %% maximize space between adjacent subfigures
  \begin{subfigure}[b]{0.48\linewidth}
    \centering
    \includegraphics[height=0.18\textheight]{remdesivir/sego_var_fig/review/2inf/dead_pop.png} 
    \caption{} 
    \label{fig:res:var_sego:dead_pop:2inf} 
  \end{subfigure} 


    \vspace{4ex}
  \begin{subfigure}[b]{0.48\linewidth}
    \centering
    \includegraphics[height=0.18\textheight]{remdesivir/sego_var_fig/review/5inf/dead_pop.png} 
    \caption{} 
    \label{fig:res:var_sego:dead_pop:5inf} 
  \end{subfigure} 
  \hspace{\fill}
   \begin{subfigure}[b]{0.48\linewidth}
    \centering
    \includegraphics[height=0.18\textheight]{remdesivir/sego_var_fig/review/10inf/dead_pop.png} 
    \caption{} 
    \label{fig:res:var_sego:dead_pop:10inf} 
  \end{subfigure} 
\caption{Dead cell populations for 400 replicas of Sego \emph{et al.}'s model~\cite{sego_modular_2020}. In all the cases the medians of simulation replicas are in black lines, the 0th to 100th quantiles are shaded as dark blue, 10th to 90th shaded in orange, and the 25th to 75th as light blue. \ref{fig:res:var_sego:dead_pop:1inf}) Simulations start with 1 initially infected cell and 7 simulations result in failure to infect (1.75\% of replicas), the 90th quantile includes the upper bound of the number of cells. \ref{fig:res:var_sego:dead_pop:2inf}) Simulations start with 2 initially infected cells where 5 simulations result in failure to infect (1.25\%), the 100th quantile includes the upper bound of the number of cells. \ref{fig:res:var_sego:dead_pop:5inf}) Simulations start with 5 initially infected cells. \ref{fig:res:var_sego:dead_pop:10inf}) Simulations start with 10 initially infected cells.}\label{fig:res:var_sego:dead_pop} 
\end{figure}



\begin{figure}[H]

  \begin{subfigure}[b]{0.48\linewidth}
    \centering
    \includegraphics[height=0.18\textheight]{remdesivir/sego_var_fig/review/1inf/viral_load.png} 
    \caption{} 
    \label{fig:res:var_sego:load:1inf} 
  \end{subfigure} 
  \hspace{\fill}  %% maximize space between adjacent subfigures
  \begin{subfigure}[b]{0.48\linewidth}
    \centering
    \includegraphics[height=0.18\textheight]{remdesivir/sego_var_fig/review/2inf/viral_load.png} 
    \caption{} 
    \label{fig:res:var_sego:load:2inf} 
  \end{subfigure} 

    \vspace{4ex}
  \begin{subfigure}[b]{0.48\linewidth}
    \centering
    \includegraphics[height=0.18\textheight]{remdesivir/sego_var_fig/review/5inf/viral_load.png} 
    \caption{} 
    \label{fig:res:var_sego:load:5inf:app} 
  \end{subfigure} 
  \hspace{\fill}
   \begin{subfigure}[b]{0.48\linewidth}
    \centering
    \includegraphics[height=0.18\textheight]{remdesivir/sego_var_fig/review/10inf/viral_load.png} 
    \caption{} 
    \label{fig:res:var_sego:load:10inf} 
  \end{subfigure} 
\caption{Extracellular viral load for 400 replicas of Sego \emph{et al.}'s model~\cite{sego_modular_2020}, y-axis in log scale. In all the cases the medians of simulation replicas are in black lines, the 0th to 100th quantiles are shaded as dark blue, 10th to 90th shaded in orange, and the 25th to 75th as light blue. \ref{fig:res:var_sego:load:1inf}) Simulations start with 1 initially infected cell and 7 simulations result in failure to infect (1.75\% of replicas), the 90th quantile includes the upper bound of the number of cells. \ref{fig:res:var_sego:load:2inf}) Simulations start with 2 initially infected cells where 5 simulations result in failure to infect (1.25\%), the 100th quantile includes the upper bound of the number of cells. \ref{fig:res:var_sego:load:5inf:app}) Simulations start with 5 initially infected cells. \ref{fig:res:var_sego:load:10inf}) Simulations start with 10 initially infected cells.}\label{fig:res:var_sego:load} 
\end{figure}

\begin{figure}[H]

  \begin{subfigure}[b]{0.48\linewidth}
    \centering
    \includegraphics[height=0.18\textheight]{remdesivir/sego_var_fig/review/1inf/viral_auc.png} 
    \caption{} 
    \label{fig:res:var_sego:auc:1inf} 
  \end{subfigure} 
  \hspace{\fill}  %% maximize space between adjacent subfigures
  \begin{subfigure}[b]{0.48\linewidth}
    \centering
    \includegraphics[height=0.18\textheight]{remdesivir/sego_var_fig/review/2inf/viral_auc.png} 
    \caption{} 
    \label{fig:res:var_sego:auc:2inf} 
  \end{subfigure} 


    \vspace{4ex}
  \begin{subfigure}[b]{0.48\linewidth}
    \centering
    \includegraphics[height=0.18\textheight]{remdesivir/sego_var_fig/review/5inf/viral_auc.png} 
    \caption{} 
    \label{fig:res:var_sego:auc:5inf} 
  \end{subfigure} 
  \hspace{\fill}
  \begin{subfigure}[b]{0.48\linewidth}
    \centering
    \includegraphics[height=0.18\textheight]{remdesivir/sego_var_fig/review/10inf/viral_auc.png} 
    \caption{} 
    \label{fig:res:var_sego:auc:10inf} 
  \end{subfigure} 
\caption{Extracellular viral AUC for 400 replicas of Sego \emph{et al.}'s model~\cite{sego_modular_2020}, y-axis in log scale. In all the cases the medians of simulation replicas are in black lines, the 0th to 100th quantiles are shaded as dark blue, 10th to 90th shaded in orange, and the 25th to 75th as light blue. \ref{fig:res:var_sego:auc:1inf}) Simulations start with 1 initially infected cell and 7 simulations result in failure to infect (1.75\% of replicas), the 90th quantile includes the upper bound of the number of cells. \ref{fig:res:var_sego:auc:2inf}) Simulations start with 2 initially infected cells where 5 simulations result in failure to infect (1.25\%), the 100th quantile includes the upper bound of the number of cells. \ref{fig:res:var_sego:auc:5inf}) Simulations start with 5 initially infected cells. \ref{fig:res:var_sego:auc:10inf}) Simulations start with 10 initially infected cells.}\label{fig:res:var_sego:auc} 
\end{figure}


