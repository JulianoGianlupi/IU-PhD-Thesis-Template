The classification of treatment outcomes into fast clearance, slow clearance, partial containment, or widespread infection is done by using quantitative metrics in an algorithm. We always use the median measurements of the simulation replicas for each parameter combination. We first look at the median time course of the uninfected population, second at the median time course of the extracellular viral load, third at the extracellular virus AUC from treatment start.

If the median uninfected population at simulation end is below ten or less than half the uninfected population at the start of treatment we classify it as ineffective with widespread infection.

The next metric we look at is median extracellular viral load. If the viral load goes below a threshold of $1.3 A.U.$ the simulation has cleared the virus at least once (there may be subsequent releases of extracellular virus) but we still don't know if the infection was contained or not. If the extracellular virus was cleared in less than 14 days of treatment and the extracellular virus concentration does not rise above a slightly higher threshold of $1.1\times1.3 A.U.= 1.43 A.U.$ after treatment initiation we classify the treatment as effective with fast containment. 

If the virus is cleared but then there is a rise above $1.43 A.U.$ we look at the maxima and minima of the logarithm of extracellular viral load post 14 days of treatment. We calculate the difference of the last maxima and the first minima ($\Delta M$) and we compare it to another threshold of $10^{-4} A.U.$ If the difference is close to zero, $|\Delta M| < 10^{-4} A.U.$, we classify the treatment as ineffective with a partial containment. If the difference is less than the negative of the threshold ($\Delta M < -10^{-4} A.U.$) we classify the treatment as effective and the clearance as slow. If the difference is above the positive threshold ($\Delta M > 10^{-4} A.U.$) we classify the result as widespread infection.

If the extracellular virus level never goes below the $1.3 A.U.$ threshold we first check against the $1.43 A.U.$ threshold, if the levels have gone below it we classify the treatment as partial containment. If not we look at $\Delta M$ (as before) and at the AUC from treatment initiation ($AUC_{TI}$). If $AUC_{TI}$ is below $300 A.U.$ we classify the treatment as ineffective with a partial containment. If not we use $\Delta M$ and the threshold of $10^{-4} A.U.$ for the classification.